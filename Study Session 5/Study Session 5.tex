\documentclass[../custom,grid]{flashcards}
%\usepackage{amsmath}
%\usepackage{booktabs}
%\usepackage{array}
\usepackage{enumitem}
%\usepackage{tikz}
%\usepackage{float}

\newcommand{\studyArea}{Portfolio Management for Institutional Investors}

\def\labelitemii{$\circ$}
\def\labelitemiii{$\diamond$}
\def\labelitemiv{$\cdot$}

\begin{document}
\cardfrontstyle{headings}
\cardfrontfoot{Study Session 5}

\begin{flashcard}[\studyArea]{General Pension Definitions}
    \begin{itemize}
        \item \textbf{Funded status} is the difference between PV of assets and liabilities.
        \item \textbf{Plan surplus} is plan assets less liabilities. Overfunded when positive, underfunded when negative, fully funded when equal.
        \item \textbf{Accumulated benifit obligation} (ABO) is total PV of liabilities to date, assuming no further benefit accumulation.
        \item \textbf{Projected benefit obligation} (PBO) is ABO plus PV of liability from projected compensation increases. Used for funded status for ongoing plans.
        \item \textbf{Total future liability} is PBO plus PV of expected increase due to current employees. Not an accounting term.
        \item \textbf{Retired lives} is total participants receiving benefits.
        \item \textbf{Active lives} is total employed participants not receiving benefits.
    \end{itemize}
\end{flashcard}

\begin{flashcard}[\studyArea]{Cash Balance and Profit Sharing Plans}
    \begin{itemize}
        \item \textbf{Cash balance plan} is a DB plan where individual balances are recorded so they are portable.
        \item \textbf{Profit sharing plan} is a DC plan where the employer contribution is based on company profits.
    \end{itemize}
\end{flashcard}

\begin{flashcard}[\studyArea]{Factors Affecting Risk Tolerance for Defined Benefit Plans}
    \begin{itemize}
        \item \textbf{Plan surplus.} Larger surplus allows more negative investment results and thus higher risk tolerance. Note that it is not acceptable to take on more risk in the event of a negative surplus. The plan sponsor and manager have an obligation to the plan beneficiaries.
        \item \textbf{Financial status and profitability.} Debt to equity and profit margins indicate financial strength. More strength implies greater risk tolerance.
        \item \textbf{Sponsor and pension fund common risk exposures.} Higher correlation between profitability and assets implies lower risk tolerance. High correlation means profitability may fall exactly when the fund's value falls.
        \item \textbf{Plan features.} Provisions for early retirement or lump-sum withdrawals decrease plan duration and decrease risk tolerance. Provisions which increase liquidity needs or reduce time horizon also decrease risk tolerance.
        \item \textbf{Workforce characteristics.} Lower workforce age increases time horizon and risk tolerance. Higher ratio of retired lives to active lives increases liquidity needs and lowers risk tolerance.
    \end{itemize}
\end{flashcard}

\begin{flashcard}[\studyArea]{Factors Affecting Objectives and Constraints for Pension Plans}
    \begin{itemize}[itemsep=.2\itemsep]
        \item \textbf{Risk and return.}
            \begin{itemize}
                \item \textbf{Future pension contributions.} Return levels can be calculated to eliminate the need for contributions to plan assets.
                \item \textbf{Pension income.} Pension expenses go on the income statement, and negative expenses are also recognized.
            \end{itemize}
        \item \textbf{Liquidity.}
            \begin{itemize}
                \item \textbf{Number of retired lives.} More retired lives compared to active participants means more liquidity is required.
                \item \textbf{Amount of sponsor contributions.} Smaller corporate contributions relative to retirement payments means more liquidity needed.
                \item \textbf{Plan features.} More liquidity for early retirement and lump-sum payouts.
            \end{itemize}
        \item \textbf{Time horizon.} If the plan is terminating, the horizon is the termination date. For ongoing plans, time horizon depends on characteristics of participants.
        \item \textbf{Taxes.} Most retirement plans are tax exempt. In some countries some portions are taxed and others are not.
        \item \textbf{Legal and regulatory factors.} The existing regulatory framework must be incorporated into the IPS@. Legal counsel is required for complex issues. A pension plan trustee or manager is a fiduciary and must act in the best interest of the participants.
    \end{itemize}
\end{flashcard}

\begin{flashcard}[\studyArea]{Risk Management Considerations for Pension Plans}
    \begin{itemize}
        \item \textit{Pension investment returns in relation to the operating returns of the company.} The company should favor low correlation with plan assets and the plan should avoid in investing in the sponsor company, in the same industry, or in companies that are highly correlated.
        \item \textit{Coordinating pension investments with pension liabilities.} Focus on managing the surplus and its stability, probability of unexpected increases in required contributions is minimized.
    \end{itemize}
\end{flashcard}
\end{document}
