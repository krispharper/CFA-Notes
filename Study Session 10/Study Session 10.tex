\documentclass[../custom]{flashcards}
\usepackage{amsmath}

\newcommand{\studyArea}{Global Bonds and Fixed Income Derivatives}

\def\labelitemii{$\circ$}
\def\labelitemiii{$\diamond$}
\def\labelitemiv{$\cdot$}

\begin{document}
\cardfrontstyle{headings}
\cardfrontfoot{Study Session 10}

\begin{flashcard}[\studyArea]{Effects of Leverage on Variability of Returns}
    \begin{itemize}
        \item As leverage increases, the variability of returns increases.
        \item As the investment return increases, the variability of returns increases.
    \end{itemize}
\end{flashcard}

\begin{flashcard}[\studyArea]{Repurchase Agreement}
    \begin{flushleft}
        The borrower (seller) agrees to repurchase it from the lender (buyer) on an agreed upon date and at an agreed upon price.\newline

        It is legally a sale and purchase, but it's effectively a collateralized loan where the difference in price is the interest on the loan. The rate of interest on the repo is called the repo rate.
    \end{flushleft}
\end{flashcard}

\begin{flashcard}[\studyArea]{Four Repo Delivery Scenarios}
    \begin{enumerate}
        \item The borrower physically delivers the collateral to the lender. This can be costly.
        \item The collateral is deposited in a custodial account at the borrower's bank. This reduces the fees of normal delivery.
        \item The transfer of securities is done electronically through banks. This is less expensive than physical delivery, but incurs fees and changes.
        \item Delivery is sometimes not required for low credit risk borrowers, familiar parties, or short-term transactions.
    \end{enumerate}
\end{flashcard}

\begin{flashcard}[\studyArea]{Factors Affecting the Repo Rate}
    \begin{itemize}
        \item Increases with the credit risk of the borrower when delivery is not required.
        \item Declines as the quality of the collateral increases.
        \item Increases with the term of the repo.
        \item Lower for physical delivery. If held by the borrower's bank, the rate will be higher. If no delivery takes place, it will be even higher.
        \item Lower for collateral with limited availability.
        \item The federal funds rate is a benchmark for repo rates. As it increases, repo rates increase.
        \item Changes with seasonal factors which change the demand for funds.
    \end{itemize}
\end{flashcard}

\begin{flashcard}[\studyArea]{Disadvantages of Standard Deviation and Variance}
    \begin{itemize}
        \item Bond returns are often not normally distributed. Bonds with options have non-normal return distributions.
        \item The number of inputs increases quadratically with portfolio size.
        \item Obtaining estimates for inputs is difficult. Historical risk measures may not be indicative of future measures. A bond's options change in influence over time.
    \end{itemize}
\end{flashcard}

\begin{flashcard}[\studyArea]{Semivariance, Shortfall Risk and Value at Risk}
    \begin{itemize}
        \item \textbf{Semivariance} measures the dispersion of returns below a target return. It is not commonly used because
            \begin{itemize}
                \item Difficult to compute for large portfolios.
                \item If the returns are symmetric, semivariance will give the same ranking as variance, which is better understood.
                \item If returns are asymmetric, downside risk can be difficult to forecast.
                \item Because it only uses half the distribution, a smaller sample size is considered, which is generally statistically less accurate.
            \end{itemize}
        \item \textbf{Shortfall risk} measures the probability that actual returns will be less than the target return. The primary disadvantage is that it doesn't account for outliers, so the magnitude of the shortfall is ignored.
        \item \textbf{Value at risk} is the probability of a return less than a given amount over a specific time period. Like shortfall risk, it doesn't provide the magnitude of losses above those specified by VaR.
    \end{itemize}
\end{flashcard}

\begin{flashcard}[\studyArea]{Cheapest to Deliver}
    \begin{flushleft}
        A bond that the counterparty in the short position can deliver to satisfy the obligation of the futures contract.\newline

        A conversion factor determines the price received at delivery. The quoted price for the CTD is the product of the quoted futures price and the conversion factor.
    \end{flushleft}
\end{flashcard}

\begin{flashcard}[\studyArea]{Advantages of Futures Over Cash Market Instruments}
    \begin{itemize}
        \item Futures are more liquid.
        \item Futures are less expensive.
        \item Futures are more easily shorted than an actual bond.
    \end{itemize}
\end{flashcard}

\begin{flashcard}[\studyArea]{How to Use Futures to Change Dollar Duration}
    \begin{flushleft}
        To increase dollar duration, buy futures contracts.\newline

        To decrease dollar duration, sell futures contracts.
    \end{flushleft}
\end{flashcard}

\begin{flashcard}[\studyArea]{Dollar Duration Hedging Formula}
    \begin{flushleft}
        The number of contracts to construct a hedge is
        \begin{align*}
            \text{number of contracts} &= \frac{\text{DD}_T - \text{DD}_P}{\text{DD}_f}
        \intertext{Using duration instead of dollar duration, we have}
            \text{number of contracts} &= \frac{(D_T - D_P) P_P}{D_\text{CTD} P_\text{CTD}} (\text{CTD conversion factor})
        \end{align*}
    \end{flushleft}
\end{flashcard}

\begin{flashcard}[\studyArea]{Basis Risk}
    \begin{flushleft}
        Price basis is the difference between the spot price and the futures price at delivery.\newline

        Basis risk is the variability of the basis. It should be considered for hedges which will end before delivery. Basis can change unexpectedly.
    \end{flushleft}
\end{flashcard}

\begin{flashcard}[\studyArea]{Cross Hedge}
    \begin{flushleft}
        The underlying security in the futures contract is not identical to the asset being hedge (e.g., using Treasury bond futures to hedge corporates). It can be either long or short. Cross hedges must be used when there is no corresponding futures contract for a position, or if such a contract is too illiquid.
    \end{flushleft}
\end{flashcard}

\begin{flashcard}[\studyArea]{Hedge Ratio for Cross Hedges}
    \begin{flushleft}
        A hedge ratio should be used when the futures contract doesn't perfectly match the underlying bond's risk factors. The ratio is
        \begin{align*}
            \text{hedge ratio} &= \frac{\text{exposure of bond to risk factor}}{\text{exposure of futures to risk factor}} \\
            &= \frac{\text{exposure of bond to risk factor}}{\text{exposure of CTD to risk factor}} \times \frac{\text{exposure of CTD to risk factor}}{\text{exposure of futures to risk factor}}
        \end{align*}
        where the last term represents the conversion factor of the CTD bond.
    \end{flushleft}
\end{flashcard}

\begin{flashcard}[\studyArea]{Yield Beta}
    \begin{flushleft}
        Yield beta is obtained from the regression equation
        \[
            \text{yield on bond} = \alpha + \beta (\text{yield on CTD}) + \varepsilon
        \]
        The yield beta, $\beta$, shows the relationship between changes in yields on the bond and the CTD.\newline

        When yield spread is not constant, we adjust the hedge ratio for interest rate risk to use the yield beta as
        \[
            \text{hedge ratio} = \frac{D_P P_P}{D_{\text{CTD}} P_{\text{CTD}}} (\text{CTD conversion factor}) (\text{yield beta})
        \]
        This implies that if $\beta > 1$ (i.e. the yield on the hedged bond is more volatile than that of the CTD), then more futures contracts will be needed to to hedge the bond.
    \end{flushleft}
\end{flashcard}

\begin{flashcard}[\studyArea]{Three Basic Sources of Hedging Error}
    \begin{enumerate}
        \item Forecast of the basis at the time the hedge is lifted.
        \item Estimated durations. Estimating duration for bonds with options can be particularly difficult.
        \item Estimated yield betas.
    \end{enumerate}
\end{flashcard}

\begin{flashcard}[\studyArea]{Formula for the Duration of a Bond Option}
    \begin{flushleft}
        \[
            \text{option delta} \times \text{duration of the underlying} \times \frac{\text{price of the underlying}}{\text{price of the option}}
        \]
    \end{flushleft}
\end{flashcard}

\begin{flashcard}[\studyArea]{Protective Put and Covered Call}
    \begin{itemize}
        \item \textbf{Protective put.} Owning a bond and buying an interest rate put. This provides protection from rising interest rates as it will compensate for the loss in value on the bond.
        \item \textbf{Covered call.} Owning a bond and selling an interest rate call. Provides income from the sale of the call, but offers no protection for rising or falling interest rates.
    \end{itemize}
\end{flashcard}

\begin{flashcard}[\studyArea]{Three Types of Credit Risk}
    \begin{itemize}
        \item \textbf{Default risk} is the risk that the issuer will not be able to pay. It is dependent on an action of the issuer.
        \item \textbf{Credit spread risk} is the risk of increase in the yield spread on an asset. It's a function of changes in the market's collective evaluation of credit quality.
        \item \textbf{Downgrade risk} is the possibility that the credit rating of an asset or issuer is downgraded by a credit-rating organization. Downgrades result in rising yields and falling prices.
    \end{itemize}
\end{flashcard}

\begin{flashcard}[\studyArea]{Three Types of Credit Derivative Instruments}
    \begin{itemize}
        \item \textbf{Credit options.} Provide protection from adverse price movements due to credit events or changes in the asset's spread. There are two types of credit options.
            \begin{itemize}
                \item \textbf{Binary credit options} are based on the underlying asset's price. A binary credit put option gives protection only after a credit event and if the value of the underlying is less than the strike price. The option value or payoff is $\text{OV} = \max ((\text{strike} - \text{value}), 0)$.
                \item \textbf{Credit spread options} are based on the underlying asset's yield spread. A credit spread call option will give protection if the reference asset's spread at maturity is above the strike spread. The option value or payoff is $\text{OV} = \max ((\text{actual spread} - \text{strike spread}) \times \text{notional} \times \text{risk factor}, 0)$.
            \end{itemize}
        \item \textbf{Credit forwards.} Forward contracts wherein the payment is a function of the credit spread over the benchmark. They are a zero-sum game. The payoff is $\text{FV} = (\text{spread at maturity} - \text{contract spread}) \times \text{notional} \times \text{risk factor}$.
        \item \textbf{Credit swaps.} E.g., credit default swaps are insurance against default on an underlying asset or issuer. One party pays a periodic fee in exchange for a commitment to pay after a credit event. The terms are custom-designed for the counterparties. They can be cash settled, or use physical delivery.
    \end{itemize}
\end{flashcard}

\begin{flashcard}[\studyArea]{Sources of Excess Return on International Bonds}
    \begin{itemize}
        \item \textbf{Market selection} involves finding opportunities for value enhancement.
        \item \textbf{Currency selection} decides how much active currency management and currency hedging a manager will use. He should not employ any currency hedging without the ability to forecast interest rate changes and their impact on exchange rates.
        \item \textbf{Duration management} is finding maturities which correspond to anticipated shifts and twists in the yield curve. Limited maturity offerings may be overcome by using fixed income derivatives.
        \item \textbf{Sector selection} involves credit analysis of entire sectors to add value. This is not an analysis of individual securities.
        \item \textbf{Credit analysis} recognizes value-added opportunities in individual securities.
        \item \textbf{Markets outside the benchmark} are corporates not included in foreign bond indicies, which are usually composed of sovereign issues.
    \end{itemize}
\end{flashcard}

\begin{flashcard}[\studyArea]{Country Beta}
    \begin{flushleft}
        A country beta or yield beta measures the relationship between changes in the foreign and domestic yields. It is given by the following regression
        \[
            \Delta \text{yield}_\text{foreign} = \beta \Delta \text{yield}_\text{domestic} + \varepsilon
        \]
        Multiplying the country beta by the change in domestic rate gives the change in foreign rate. Multiplying that by the bond's duration gives the estimated change in price.
    \end{flushleft}
\end{flashcard}

\begin{flashcard}[\studyArea]{Interest Rate Parity}
    \begin{flushleft}
        \begin{align*}
            F &= S_0 \left ( \frac{1 + c_d}{1 + c_f} \right )\\
            \\
            \text{where:}\\
            F &= \text{the forward exchange rate (domestic per foreign)}\\
            S_0 &= \text{the current spot exchange rate (domestic per foreign)}\\
            c_d &= \text{the domestic short-term rate}\\
            c_f &= \text{the foreign short-term rate}
        \end{align*}
    \end{flushleft}
\end{flashcard}

\begin{flashcard}[\studyArea]{Methods of Hedging Currency Risk for International Bonds}
    \begin{itemize}
        \item \textbf{Forward hedge.} Used to eliminate most of the currency risk. Assuming forward contracts are available and actively traded, a manager enters a forward contract to sell the foreign currency at the current forward rate.
        \item \textbf{Proxy hedge.} Enter into a forward contract between the domestic currency and a second foreign currency that is correlated with the first. Proxy hedges are used when forward contracts on the first currency are not actively traded or if hedging the first currency is expensive.
        \item \textbf{Cross hedge.} Enter into a contract to deliver the original foreign currency for a third currency. This effectively eliminates the currency risk of the bond by replacing it with the risk of another currency. This doesn't eliminate the risk exposure, just changes it.
    \end{itemize}
\end{flashcard}

\begin{flashcard}[\studyArea]{Advantages and Disadvantages of Investing in Emerging Market Debt}
    \begin{itemize}
        \item Advantages
            \begin{itemize}
                \item Generally provides diversification.
                \item Increased quality in emerging market sovereign bonds.
                \item Increased resiliency. Emerging markets tend to bounce back faster.
                \item An undiversifed index like the EMBI+ offers return-enhancing potential.
            \end{itemize}
        \item Disadvantages
            \begin{itemize}
                \item Unlike emerging market governments, corporations are not able to offset negative events.
                \item Returns can be highly volatile with negatively skewed distributions.
                \item Lack of transparency and regulations means higher credit risk.
                \item Lax legal systems which don't protect against government actions.
                \item Lack of standardized covenants forces managers to carefully study each issue.
                \item Political risk including instability, changes in taxation and regulation, currency exchange restrictions, relaxed bankruptcy regulations, imposed exchange rate changes.
                \item Lack of diversification of the EMBI+ which is concentrated in Latin America.
            \end{itemize}
    \end{itemize}
\end{flashcard}

\begin{flashcard}[\studyArea]{Criteria to Determine the Optimal Mix of Active Managers}
    \begin{itemize}
        \item \textbf{Style analysis.} Most active returns can be explained by selected style. Primary concerns are not just the style employed, but also risk factors associated with the style.
        \item \textbf{Selection bets.} Includes credit spread analysis and the identification of over- and under-valued securities.
        \item \textbf{Investment process.} Investigate all aspects of the investment process including types of research performed, how alpha is attained, and if decisions are made by committee or individual.
        \item \textbf{Alpha correlations.} Alphas should be diversified so that alphas of various managers are not all correlated. This reduces volatility of alpha as well as the chance that all alphas will be negative at the same time.
    \end{itemize}
\end{flashcard}

\end{document}
