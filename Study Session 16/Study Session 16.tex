\documentclass[../custom,grid]{flashcards}
\usepackage{amsmath}
%\usepackage{booktabs}
%\usepackage{array}
%\usepackage{enumitem}
%\usepackage{tikz}
%\usepackage{float}

\newcommand{\studyArea}{Trading, Monitoring and Rebalancing}

\def\labelitemii{$\circ$}
\def\labelitemiii{$\diamond$}
\def\labelitemiv{$\cdot$}

\begin{document}
\cardfrontstyle{headings}
\cardfrontfoot{Study Session 16}

\begin{flashcard}[\studyArea]{Market and Limit Orders}
    \begin{itemize}
        \item \textbf{Market orders} execute immediately at the best possible price. May be filled by multiple trades. The advantage is the execution speed. Disadvantage is that the price isn't known ahead of time. Thus it has price uncertainty.
        \item \textbf{Limit orders} execute at the limit price or better. Can be set to expire after a set amount of time, but if prices don't move correctly, it may not execute. Thus it has execution uncertainty.
    \end{itemize}
\end{flashcard}

\begin{flashcard}[\studyArea]{Bid-Ask and Effective Spreads}
    \begin{itemize}
        \item \textbf{Limit order book} is a dealer's offering of securities.
        \item \textbf{Inside bid} or \textbf{market bid} is the best bid price.
        \item \textbf{Inside ask} or \textbf{market ask} is the best ask price.
        \item \textbf{Inside bid-ask spread} or \textbf{market bid-ask spread} is the difference between the inside bid and ask.
        \item \textbf{Midquote} is the average of the inside bid and ask.
        \item \textbf{Effective spread} is a measure of the round trip cost of a transaction. It reflects price improvement, trades executed at better than the bid-ask quote, and price impact, trades executed at worse than the bid-ask quote.
            \begin{align*}
                \text{effective spread for a buy} &= 2 \times (\text{execution price} - \text{midquote})\\
                \text{effective spread for a sell} &= 2 \times (\text{midquote} - \text{execution price})\\
            \end{align*}
    \end{itemize}
\end{flashcard}

\begin{flashcard}[\studyArea]{Types of Markets Structures}
    \begin{itemize}
        \item \textbf{Quote-driven markets.} Traders interact with dealers who provide liquidity by willingness to buy or sell. Closed-book markets require a broker to interact.
        \item \textbf{Order-driven markets.} Traders interact with other traders. Prices are set by supply and demand. Disadvantage is that liquidity may be poor. Execution is determined by a mechanical rule. In an electronic crossing network, traders are institutions and are anonymous. In an auction market, orders compete against others. Can be periodic (batch) or continuous. Automated auctions trade continuously and execute based on a set of rules. They are like ECNs but with price discovery.
        \item \textbf{Brokered markets.} Brokers act as traders' agents to find counterparties for the traders.
    \end{itemize}
\end{flashcard}

\begin{flashcard}[\studyArea]{Roles of Brokers}
    \begin{itemize}
        \item \textbf{Acts as a trader's agent} which imposes a legal obligation to act in his best interest.
        \item \textbf{Represent the order} and advise the trader on price and volume for execution.
        \item \textbf{Find counterparties} via contacts, market information, or by acting as a dealer.
        \item \textbf{Provide secrecy} if the trader wishes to remain anonymous.
        \item \textbf{Provide other services} such as record keeping, safe keeping, or cash management. Not liquidity, which is the role of the dealer.
        \item \textbf{Support the market} indirectly by participating.
    \end{itemize}
\end{flashcard}

\begin{flashcard}[\studyArea]{Components of Market Quality}
    \begin{itemize}
        \item \textbf{Liquidity} is measured by small bid-ask spreads, market depth allowing larger orders, and resilience providing price accuracy. Factors needed for a liquid market are
            \begin{itemize}
                \item Many buyers and sellers so traders can reverse positions if necessary.
                \item Diverse investors with different information and opinions.
                \item Convenient location or trading platform.
                \item Integrity determined by participants and regulation so all traders are treated fairly.
            \end{itemize}
        \item \textbf{Transparency} means traders can get pre- (quotes and spreads) and post- (completed trades) trade information quickly and cheaply.
        \item \textbf{Assurity of completion} means traders have confidence counterparties will uphold their side of the trade. Brokers and clearinghouses may provide guarantees to this end.
    \end{itemize}
\end{flashcard}
\end{document}
